\documentclass[conference]{IEEEtran}

\title{
  Course ID2210\\
  Project Report
}

\author{
  \IEEEauthorblockN{Riccardo Reale}
  \IEEEauthorblockA{Peerialism AB\\
    {riccardo.reale@peerialism.com}\\
    \url{https://github.com/riccardoreale/id2210-vt14.git}
  }
  \and
  \IEEEauthorblockN{Giovanni Simoni}
  \IEEEauthorblockA{Peerialism AB\\
    {giovanni.simoni@peerialism.com}\\
    \url{https://github.com/dacav/id2210-vt14.git}
  }
}

\usepackage{xspace}
\usepackage{hyperref}
\usepackage{braket}
\usepackage{graphicx}
\usepackage{amssymb}
\usepackage{amsmath}
\usepackage{amsthm}
\usepackage{booktabs}
\usepackage{multirow}
\usepackage{paralist}

\usepackage[ruled,vlined,nofillcomment,linesnumbered]{algorithm2e}

\newcommand{\actor}[1]{\textsc{#1}\xspace}
\newcommand{\kword}[1]{\textsc{#1}\xspace}
\newcommand{\component}[1]{\texttt{#1}\xspace}

\newcommand{\us}{\textit{Sparrow}\xspace}
\newcommand{\omni}{\textit{Omniscient}\xspace}

% Actors
\newcommand{\dc}{\actor{DataCenter}}
\newcommand{\tmast}{\actor{Task Master}}
\newcommand{\exc}{\actor{Executor}}

% Components
\newcommand{\RmWorker}{\component{RmWorker}}
\newcommand{\ResourceManager}{\component{ResourceManager}}

\newcommand{\treq}{\kword{requisites}}
\newcommand{\Treq}{\kword{Requisites}}
\newcommand{\capable}{\kword{capable}}
\newcommand{\Capable}{\kword{Capable}}



\begin{document}
\maketitle

\section{Architecture design}

  We implemented two different behavioral modes. The first implements a
  modified version of the scheduling logic illustrated by the paper,
  henceforth referred as \us, while the second implements
  an omniscent scheduler for sake of comparison with an optimal assignment
  strategy, henceforth referred as \omni.

  The required number of CPUs and amount of memory for the execution of a
  task will be referred as \treq. \Treq are allocated by the node executing the
  task (\exc) on behalf of the assigner (\tmast). The allocation time is
  considered as the payload of a task assignment, while the execution
  consists in allocating the resources (according to \treq), waiting the
  specified amount of time, and releasing the resources.

  Here follows the work-flow for a task assignment in \us:
  \begin{enumerate}

  \item The task gets issued by the \dc and propagated
        to a \tmast, which is selected as the closest to the task
        identifier within a consistent hash table;

  \item If the \treq can be satisfied by the \tmast, the execution is
        achieved immediately (namely the \tmast becomes the \exc),
        otherwise probing messages are sent to a number $P$ of selected
        nodes.

  \end{enumerate}

  \subsection{Selection of nodes to be probed}

\section{Quick class reference}

\section{Experimental evaluation}

\end{document}
