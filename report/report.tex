\documentclass[conference]{IEEEtran}

\title{
  Course ID2210\\
  Project Report
}

\author{
  \IEEEauthorblockN{Riccardo Reale}
  \IEEEauthorblockA{Peerialism AB\\
    {riccardo.reale@peerialism.com}\\
    \url{https://github.com/riccardoreale/id2210-vt14.git}
  }
  \and
  \IEEEauthorblockN{Giovanni Simoni}
  \IEEEauthorblockA{Peerialism AB\\
    {giovanni.simoni@peerialism.com}\\
    \url{https://github.com/dacav/id2210-vt14.git}
  }
}

\usepackage{xspace}
\usepackage{hyperref}
\usepackage{braket}
\usepackage{graphicx}
\usepackage{amssymb}
\usepackage{amsmath}
\usepackage{amsthm}
\usepackage{booktabs}
\usepackage{multirow}
\usepackage{paralist}

\usepackage[ruled,vlined,nofillcomment,linesnumbered]{algorithm2e}

\newcommand{\actor}[1]{\textsc{#1}\xspace}
\newcommand{\kword}[1]{\textsc{#1}\xspace}
\newcommand{\component}[1]{\texttt{#1}\xspace}

\newcommand{\us}{\textit{Sparrow}\xspace}
\newcommand{\omni}{\textit{Omniscient}\xspace}

% Actors
\newcommand{\dc}{\actor{DataCenter}}
\newcommand{\tmast}{\actor{Task Master}}
\newcommand{\exc}{\actor{Executor}}

% Components
\newcommand{\RmWorker}{\component{RmWorker}}
\newcommand{\ResourceManager}{\component{ResourceManager}}

\newcommand{\treq}{\kword{requisites}}
\newcommand{\Treq}{\kword{Requisites}}
\newcommand{\capable}{\kword{capable}}
\newcommand{\Capable}{\kword{Capable}}



\begin{document}
\maketitle

\section{Architecture design}

  \subsection{Base Sparrow}

  The base system was implemented following the suggested design with the
  addition of two main components. Each simulated Peer is composed by:
  \begin{enumerate}

  \item A \ResourceManager component, which abstracts the task assignment
    operations on the network, implementing the selection and the
    signaling through the network.

  \item A \RmWorker component, which implements a FIFO queue for the task and
    handles the simulated execution logic.

  \item A \component{FailureDetector} component, which detects not
    responding nodes that are subscribed to be monitored. If a monitored
    node doesn't answer to pings after a certain timeout (3 seconds),
    the Detector will assume the node to have crashed. For simplicity we
    assume that there are no network failures in the Data Center, so if a
    node is detected as dead, it will stay so.

  \end{enumerate}

  The required number of CPUs and amount of memory for the execution of a
  task will be referred as \treq. \Treq are allocated by the node
  executing the task (\exc) on behalf of the assigner (\tmast). The
  allocation time is considered as the payload of a task assignment and
  we assume is not known a priori by a Resource Manager. Therefore the
  allocation time is used only to simulate the task execution, which
  consists in allocating the resources (according to \treq), waiting the
  specified amount of time, and releasing the resources.

  We implemented two different behavioral modes. The firdt implements the
  best scheduling logic illustrated by the Sparrow paper (Batch + Late
  Binding), henceforth referred as \us, while the second implements an
  omniscient scheduler for sake of comparison with an optimal assignment
  strategy, henceforth referred as \omni. This is obtained directly in the
  \dc scheduler, which, for each task, will calculate the first
  node to have the requested resources available, and assign it to be
  executed directly on that node as soon as possible. This operation is,
  of course, unrealistic since it make use of both updated information
  about the task queue on each node, and the allocation time of each task.

  The work-flow for \us algorithm is the follow:

  \begin{enumerate}

    \item The task gets issued by the \dc and propagated to a \tmast,
      which is selected as the closest to the task identifier within a
      consistent hash table;

    \item The \tmast selects a number of his neighbours as candidates for
      the role of \exc. The task is kept in hold while the neighbours get
      signaled with a probe. The probe indicates the \treq for the task.

    \item When the \ResourceManager of a node receives a probe, it
      forwards the request to the \RmWorker component, which put a
      place-holder in their own waiting queue. The operation achievement
      is notified to the \tmast.

    \item When \RmWorker enqueues a new task place-holder the next task
      the waiting queue is considered for execution. If enough resources
      are available, the \treq resources are pre-allocated, and a
      confirmation request is forwarded through the \ResourceManager
      component, to the \tmast.

    \item When a \tmast receives a request for confirmation from an \exc,
      it send an assignment to him with the oldest not-yet-assigned task.
      The assigned task may differ from the one originally referred by the
      probe, as long as characterized by smaller or equal \treq. If
      no tasks can be assigned, either because there's no task in hold, or
      because they exceed the \treq of the original one, the \exc is
      notified with a cancellation signal.

    \item If an \exc waiting for a confirmation receives an assignment,
      its \RmWorker updates executes the task, possibly freeing part of
      the pre-allocated resources if the \treq declared in the
      place-holder differ from the actual task.

    \item If an \exc waiting for a confirmation receives a cancellation
      instead, its \RmWorker component simply de-allocates the temporary
      blocked resources, and executes 4.

    \item When the \RmWorker of the \exc completes a task execution, it
      releases the blocked resources, notifies its \ResourceManager and
      behaves as in point 4.

  \end{enumerate}

 Concerning Fault Tolerance, the ResourceManager subscribes to its FailureDetector each node it probes and check that every probe is received. If a node is detected as dead, the \tmast will restart the probe process for each task that was running on the dead node. Similarly the RmWorker of an \exc subscribes each \tmast. If a \tmast is detected as dead, the RmWorker will cancel all his tasks.
  

  \subsection{Improved Sparrow}
  The improved version of Sparrow consists in four main algorithm modifications:
   \begin{enumerate}

  \item \textbf{Self Assignment }- A ResourceManager receiving a Task from the DataCenter will directly execute it if its RmWorker has enough resources immediately available.
  
  \item \textbf{Probe Propagation} - A ResourceManager receiving a probe can decide to deposit it on its RmWorker or propagating it to another neighbour.  The probe will be locally deposited if its task can be executed immediately or it has been propagated more than a fixed number of time (usually 5).
  
   \item \textbf{Next Hop Selection} - When probing or propagating a probe, the peer selected from the neighbours can be chosen randomly (Random Walks) or in a greedy way (Greedy Search), based on the current knowledge of his resources availability provided by Cyclon We used a SoftMax algorithm with configurable temperature.
  
  \item \textbf{Gradient View} - Instead of using the the neighbors view provided by Cyclon, the RmWorker can use the view provided by Gradient, implemented using a Gradient Ranking function on top of TMan, based on the available node resources. We used a greedy ranking function, while the utility function is very basic and defined as:
\begin{equation}
 U_p(n) = CPU_p(n) - Queue_p(n)
\end{equation}
Where $CPU_p(n)$ means the available CPU of the node \textit{n} according to the value cached by node\textit{ p}, and $Queue_p(n)$ the length of its waiting queue.
   The reason of this choice will be clarified in the Evaluation.
  
  \end{enumerate}

\section{Experimental evaluation}

\subsection{Setup}
To evaluate our improvement over Sparrow we selected two main scenarios:
\begin{itemize}

\item A load scenario where 50 homogeneous nodes (8 cpus and 16GB memory) executes 1000 tasks, each requesting 2 cpus and 2Gb of memory and a fixed execution time of 10 seconds. The load simply determines the inter-arrival time of the tasks, based on the number of total cpu in the DataCenter (the dominant resource) and the processing time.

\item A random scenario (\textbf{A}) which consists in using  a variable number of homogeneous nodes (8 cpus and 16GB memory), executing 500 tasks with different requirements in terms of only cpu (between 1 and 8) and time (between 1 and 30 seconds), which arrive at a rate of half second between each.

\item A random scenario (\textbf{B}) similar to A but tasks have random values of both cpu and memory (between 1 and 8 GB). For this particular scenario we changed the utility function to include also the memory.

\item A random scenario (\textbf{C}) similar to A but with heterogeneous nodes resources.

\end{itemize}

Except for scenario \textit{B}, all of them are generated using the CPU as dominant resource, for easily controlling the load and the greedy ranking functions, both for Greedy Search and Gradient.

We added and evaluated the different algorithm improvements adding them to Sparrow base one to obtain three different configurations: 
\begin{enumerate}
\item \textbf{Random Walk} uses Self Assignment and Probe Propagation, with random Next Hop Selection (Temperature set to 100), giving us Random Walks for each probe. 
\item \textbf{Greedy Search} adds Greedy Next Hop Selection.
\item \textbf{Gradient Search} uses Greedy Next Hop Selection on top of Gradient View.
\end{enumerate}

As main measures we use, as suggested, the average queue time of each task and the 99th percentile.
Each result is average over 5 runs with different seed.

\subsection{Results in Load Scenario}

Figure \ref{fig:probes} shows the base implementation of Sparrow in the Load scenario, for different number of probes used. The results are consistent with the one presented in the Sparrow work.

Figure \ref{fig:comparison} instead shows the results comparison between the Base Sparrow algorithm (2 probes) and the improved configurations.

\begin{figure}
\begin{center}
\includegraphics[scale=0.5]{figures/probes_new}
\caption{Base Sparrow algorithm using different load and number of probes}
\label{fig:probes}
\end{center}
\end{figure}

\begin{figure}
\begin{center}
\includegraphics[scale=0.5]{figures/comparison_new}
\caption{Comparison between Base Sparrow Algorithm and its enhancements}
\label{fig:comparison}
\end{center}
\end{figure}

The comparison shows how the Self Assignment decreases the average queue time, since in most cases the task won't be propagated at all but immediately executed on the \tmast.

The Probe Propagation instead decreases dramatically the 99th percentile queue time, especially with loads between 50\% and 85\%, and similarly on all the configurations. Above 85\%, all configurations experience an increase in both average and 99th percentile. 

Using a Greedy Search on top of Random View doesn't improve significantly neither average or 99th percentile.

Finally the last configuration, using Gradient, performs slightly worse than the others. At higher loads probably Cyclon is not fast enough to provide very fresh samples of the neighbour resources, which results on the nodes on top of the gradient to receive an higher amount of probes. 

The Gradient Ranking function is not too useful when the load increases too much since the use of Late Binding provides a poor correlation between waiting queue length and actual load, since any of the place-holder could get a Cancel on confirmation request.

\subsection{Results in Random Scenarios}

\begin{figure}
\begin{center}
\includegraphics[scale=0.45]{figures/randomA}
\caption{Random Scenario A}
\label{fig:comparison}
\end{center}
\end{figure}

\begin{figure}
\begin{center}
\includegraphics[scale=0.45]{figures/randomB}
\caption{Random Scenario B}
\label{fig:comparison}
\end{center}
\end{figure}

\begin{figure}
\begin{center}
\includegraphics[scale=0.45]{figures/randomC}
\caption{Random Scenario C}
\label{fig:comparison}
\end{center}
\end{figure}

The Random Scenarios show a similar trend with the load scenario, except for the scenario C, with heterogeneous nodes, where Greedy Search helps to decrease the queue time in the worst cases.
This is therefore the only case, in our implementation, where a Greedy Search or Gradient gives slightly better result. In case of node heterogeneity, the random walks are sufficient to balance the load among all nodes

\end{document}
